
\documentclass[paper=a4, fontsize=11pt]{scrartcl} % A4 paper and 11pt font size
\usepackage{physics}
\usepackage[T1]{fontenc} % Use 8-bit encoding that has 256 glyphs
\usepackage{fourier} % Use the Adobe Utopia font for the document - comment this line to return to the LaTeX default
\usepackage[english]{babel} % English language/hyphenation
\usepackage{amsmath,amsfonts,amsthm} % Math packages
\usepackage{braket}
\usepackage{lipsum} % Used for inserting dummy 'Lorem ipsum' text into the template
\usepackage{tikz}
\usepackage{amsmath}
\usepackage{sectsty} % Allows customizing section commands
\allsectionsfont{\centering \normalfont\scshape} % Make all sections centered, the default font and small caps

\usepackage{bm}
\newcommand{\uvec}[1]{\boldsymbol{\hat{\textbf{#1}}}}
\usepackage[thinlines]{easytable}
\usepackage{fancyhdr} % Custom headers and footers
\pagestyle{fancyplain} % Makes all pages in the document conform to the custom headers and footers
\fancyhead{} % No page header - if you want one, create it in the same way as the footers below
\fancyfoot[L]{} % Empty left footer
\fancyfoot[C]{} % Empty center footer
\fancyfoot[R]{\thepage} % Page numbering for right footer
\renewcommand{\headrulewidth}{0pt} % Remove header underlines
\renewcommand{\footrulewidth}{0pt} % Remove footer underlines
\setlength{\headheight}{13.6pt} % Customize the height of the header
\usepackage{float}
\numberwithin{equation}{section} % Number equations within sections (i.e. 1.1, 1.2, 2.1, 2.2 instead of 1, 2, 3, 4)
\numberwithin{figure}{section} % Number figures within sections (i.e. 1.1, 1.2, 2.1, 2.2 instead of 1, 2, 3, 4)
\numberwithin{table}{section} % Number tables within sections (i.e. 1.1, 1.2, 2.1, 2.2 instead of 1, 2, 3, 4)

\setlength\parindent{0pt} % Removes all indentation from paragraphs - comment this line for an assignment with lots of text

%----------------------------------------------------------------------------------------
%	TITLE SECTION
%----------------------------------------------------------------------------------------

\newcommand{\horrule}[1]{\rule{\linewidth}{#1}} % Create horizontal rule command with 1 argument of height

\title{	
\normalfont \normalsize 
\textsc{California State University San Marcos \\ Dr. De Leone, Physics 323} \\ [25pt] % Your university, school and/or department name(s)
\horrule{0.5pt} \\[0.4cm] % Thin top horizontal rule
\huge H.W. II \\ % The assignment title
\horrule{2pt} \\[0.5cm] % Thick bottom horizontal rule
}

\author{Josh Lucas} % Your name

\date{\normalsize\today} % Today's date or a custom date

\begin{document}

\maketitle % Print the title

%----------------------------------------------------------------------------------------
%	PROBLEM 1
%----------------------------------------------------------------------------------------

\section{SPINS Lab 1}
This lab entailed preparing particles in a particular state, $\ket{\pm}_n$, and passing them through a second analyzer in the orthogonal directions noting the output on the table.
\renewcommand{\arraystretch}{2}
\begin{center}
  \begin{tabular}{ | c | c | c | c | c | c | c | }
    \hline
    $|\bra{out}\ket{in}|^2$ & $\ket{+}$& $\ket{-}$ & $\ket{+}_x$ & $\ket{-}_x$ & $\ket{+}_y$ & $\ket{-}_y$ \\ \hline
  $  \bra{+}$ & 1 &0   & $\tfrac{1}{2}$ & $\tfrac{1}{2}$  & $\tfrac{1}{2}$ & $\tfrac{1}{2}$   \\ \hline
    $\bra{-}$ & 0 & 1 & $\tfrac{1}{2}$ &$\tfrac{1}{2}$  & $\tfrac{1}{2}$ &  $\tfrac{1}{2}$ \\
    \hline
       $_x\bra{+}$ & $\tfrac{1}{2}$ & $\tfrac{1}{2}$ & 1 & 0 & $\tfrac{1}{2}$ & $\tfrac{1}{2}$  \\
    \hline
       $_x\bra{-}$ & $\tfrac{1}{2}$ & $\tfrac{1}{2}$ & 0 & 1 &$\tfrac{1}{2}$  & $\tfrac{1}{2}$  \\
    \hline
       $_y\bra{+}$ & $\tfrac{1}{2}$ &$\tfrac{1}{2}$  &$\tfrac{1}{2}$  &$\tfrac{1}{2}$  & 1 &  0 \\
    \hline
       $_y\bra{-}$ &$\tfrac{1}{2}$  &$\tfrac{1}{2}$  & $\tfrac{1}{2}$ & $\tfrac{1}{2}$ & 0 & 1  \\
    \hline
  \end{tabular}
\end{center}
We can see the symmetry in the results of the table, if the in's and out's were reversed the results would be the same.
%------------------------------------------------
\section{Ch.1 Definitions}
\textbf{Orbital Angular Momentum L} $\bullet$ The electron, from a classical viewpoint has angular momentum around a radius. $L = mvr$ This momentum of a charged particle creates current loops which induce a magnetic moment.\\
\\
 \textbf{Intrinsic Angular Momentum, S, Spin} $\bullet$ Similar to a satellite in orbit rotating as it revolves an electron classically has a rotation called spin. The spin also produces current loops.\\
\\ 
 \textbf{Gyroscopic Ratio} $\bullet$ A dimensionless scalar constant that relates spin of a particle with the magnetic moment $\mu$ as in $\vec{\mu} = g\frac{q}{2m}\vec{S}$.\\
\\
\textbf{Quantization} $\bullet$ The act of organizing information into discrete sets.\\
\\
\textbf{Spin $\frac{1}{2}$ system} $\bullet$ The Stern-Gerlach observations showed that spins of the particles were orientated in either a up or down configuration, $\vec{S_z}= \pm \frac{\hbar}{2} $\\
\\ 
 \textbf{Observable} $\bullet$ The physical quantity that is measured.\\
 \\
 \textbf{Analyzer} $\bullet$ A device which can distinguish the spin configuration of particles.\\
 \\
 \textbf{Ket} $\bullet$ Symbolic representation of the quantum state.\\
 \\
 \textbf{Bra} $\bullet$ Symbolic representation of the complex conjugate of quantum state.\\
 \\
 \textbf{State Preparation Device} $\bullet$ When a device such as Stern-Gerlach device separates the up and down spins of particles they are said to be prepared in that state. \\
 \\
 \textbf{Local Hidden Variable Theory} $\bullet$ The theory that there could be another yet undiscovered fact or trait about the system that would account for the observations.\\
 \\
 \textbf{Incompatible Observables} $\bullet$  Observables that cannot be know simultaneously as in knowing orthogonal spin components of a particle.\\
 \\
 \textbf{Quantum State Vectors} $\bullet$ The mathematical representation of the quantum state of a system containing both the real and complex information. \\
 \\
\textbf{Hilbert Space} $\bullet$ Vector space defined for a system that limits space to the necessary dimensions.\\
 \\
 \textbf{Dot Product, inner product, scalar product} $\bullet$ The magnitude of two vectors multiplied by the cosine of the angle between them. It is the magnitude of the projection of one vector on to another. \\
 \\
 \textbf{$S_z$ basis} $\bullet$ The possible contents of the set.\\
 \\
 \textbf{Orthonormality} $\bullet$ The combination of normalizing and making orthogonal a vector or state.\\
 \textbf{Basis Vectors :}\\
 \textbf{Normalized} $\bullet$ The spatial vector is normalized if it is a unit vector with magnitude of one.
 \begin{equation*}
  \hat{i}\cdot\hat{i}=\hat{j}\cdot\hat{j}=\hat{k}\cdot\hat{k} = 1
 \end{equation*}
 \textbf{Orthogonal} $\bullet$ The spatial vectors are orthogonal if their directions are perpendicular to each other.
 \begin{equation*}
 \hat{i}\cdot\hat{j}=\hat{i}\cdot\hat{k}=\hat{j}\cdot\hat{k} = 0
 \end{equation*}
 \textbf{Completeness} $\bullet$ The spatial vectors are complete if any vector to some point in the space can be written as linear superposition of the unit vectors, scaling them accordingly and summing directions.
 \begin{equation*}
 \vec{A} = a_x\hat{i} = a_y\hat{j} = a_z\hat{k} 
 \end{equation*}
\textbf{Normalization Constant} $\bullet$  The complex constant C\\
 \\
 \textbf{Probability} $\bullet$ Mathematical possibility of an event occurring, the chance that the quantum state of the ket is measured to be in the corresponding basis state.\\
 \\
 \textbf{Probability Amplitude} $\bullet$ The inner product of input and output states. \\
 \\
 \textbf{Superposition State, Coherent Superposition} $\bullet$ The combination of general spin 1/2 vector containing both up and down kets\\
 \\
 \textbf{Mixed State} $\bullet$ A state created from a mixture of up and down spin atoms \\
\\
 \textbf{Column-Row Vectors} $\bullet$ Matrices for indexing information, framework for linear algebra or computational calculations. \\
 \\
 \textbf{Representation} $\bullet$ Collection of coefficients that multiply the basis set\\
 \\
 \textbf{Kronecker Delta} $\bullet$ Binary logical operator, True if equal, false if not.  \\
 \\
 
 
 
%----------------------------------------------------------------------------------------

\end{document}