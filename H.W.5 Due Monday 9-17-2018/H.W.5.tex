
\documentclass[paper=a4, fontsize=11pt]{scrartcl} % A4 paper and 11pt font size
\usepackage{physics}
\usepackage[T1]{fontenc} % Use 8-bit encoding that has 256 glyphs
\usepackage{fourier} % Use the Adobe Utopia font for the document - comment this line to return to the LaTeX default
\usepackage[english]{babel} % English language/hyphenation
\usepackage{amsmath,amsfonts,amsthm} % Math packages
\usepackage{braket}
\usepackage{lipsum} % Used for inserting dummy 'Lorem ipsum' text into the template
\usepackage{tikz}
\usepackage{amsmath}
\usepackage{sectsty} % Allows customizing section commands
\allsectionsfont{\centering \normalfont\scshape} % Make all sections centered, the default font and small caps
\usepackage[mathscr]{euscript}
\usepackage{bm}
\newcommand{\uvec}[1]{\boldsymbol{\hat{\textbf{#1}}}}
\usepackage[thinlines]{easytable}
\usepackage{fancyhdr} % Custom headers and footers
\pagestyle{fancyplain} % Makes all pages in the document conform to the custom headers and footers
\fancyhead{} % No page header - if you want one, create it in the same way as the footers below

\usepackage{multicol}
\fancyfoot[L]{} % Empty left footer
\fancyfoot[C]{} % Empty center footer
\fancyfoot[R]{\thepage} % Page numbering for right footer
\renewcommand{\headrulewidth}{0pt} % Remove header underlines
\renewcommand{\footrulewidth}{0pt} % Remove footer underlines
\setlength{\headheight}{13.6pt} % Customize the height of the header
\usepackage{float}
\numberwithin{equation}{section} % Number equations within sections (i.e. 1.1, 1.2, 2.1, 2.2 instead of 1, 2, 3, 4)
\numberwithin{figure}{section} % Number figures within sections (i.e. 1.1, 1.2, 2.1, 2.2 instead of 1, 2, 3, 4)
\numberwithin{table}{section} % Number tables within sections (i.e. 1.1, 1.2, 2.1, 2.2 instead of 1, 2, 3, 4)

\setlength\parindent{0pt} % Removes all indentation from paragraphs - comment this line for an assignment with lots of text
\usepackage{pgfplots}

\pgfplotsset{
  compat=newest,
  xlabel near ticks,
  ylabel near ticks
}
%----------------------------------------------------------------------------------------
%	TITLE SECTION
%----------------------------------------------------------------------------------------

\newcommand{\horrule}[1]{\rule{\linewidth}{#1}} % Create horizontal rule command with 1 argument of height

\title{	
\normalfont \normalsize 
\textsc{California State University San Marcos \\ Dr. De Leone, Physics 323} \\ [25pt] % Your university, school and/or department name(s)
\horrule{0.5pt} \\[0.4cm] % Thin top horizontal rule
\huge H.W. 5 \\ % The assignment title
\horrule{2pt} \\[0.5cm] % Thick bottom horizontal rule
}

\author{Josh Lucas} % Your name

\date{\normalsize\today} % Today's date or a custom date

\begin{document}

\maketitle % Print the title

%----------------------------------------------------------------------------------------
%	PROBLEM 1
%----------------------------------------------------------------------------------------

\section*{Problem 1.13}
\textbf{Consider a quantum system with an observable A that has three possible measurement
results: a1, a2, and a3.}\\
\textbf{a) Write down the three kets $\ket{a_1}, \ket{a_2}, \ket{a_3},$ corresponding to these possible results
using matrix notation.}\\
\\
We can write the state vectors in matrix notation by choosing  orthogonal directions for each observable, giving them a magnitude of one and writing them in column form.
$$
\ket{a_1} =
\begin{pmatrix}
1\\ 0\\0
\end{pmatrix},
\quad
\ket{a_2} =
\begin{pmatrix}
0\\ 1\\0
\end{pmatrix},
\quad
\ket{a_3} = 
\begin{pmatrix}
0\\0\\1
\end{pmatrix}
$$
\textbf{b) The system is prepared in the state}\\
$$\ket{\psi} = 1\ket{a_1} -2\ket{a_2} +5\ket{a_3}$$
\textbf{Write this state in matrix notation and calculate the probabilities of all possible measurement
results of the observable A. Plot a histogram of the predicted measurement results.}\\
\\
The magnitude of each observable is multiplied by its unit vector.
\begin{equation*}
\ket{\psi} = \begin{pmatrix}
1\\0\\0
\end{pmatrix} -
2 \begin{pmatrix}
0\\1\\0
\end{pmatrix} +
5 \begin{pmatrix}
0\\0\\1
\end{pmatrix}\ =\
\begin{pmatrix}
1\\-2\\5
\end{pmatrix}
\end{equation*}
We need to normalize the wave function by dividing the ket by its magnitude,
\begin{align*}
\ket{\psi} & = C
\begin{pmatrix}
1\\-2\\5
\end{pmatrix} \\
\braket{\psi|\psi} & = C^2 (\ 1\quad -2\quad 5\ ) 
\begin{pmatrix}
1\\-2\\5
\end{pmatrix}\\
& = C^2(1+4+25) \\
C & = \frac{1}{\sqrt{30}}\\
\ket{\psi} & = \frac{1}{\sqrt{30}} 
\begin{pmatrix}
1\\-2\\5
\end{pmatrix}
\end{align*}
We can find the probabilities of each result by multiplying possible bra measurement direction with the normalized ket, 
\begin{multicols}{3}
\noindent
\begin{align*}
\mathscr{P}_{a_1} & = \bigg| \braket{a_1|\psi} \bigg|^2 \\
& = \Bigg| (\ 1\quad 0\quad 0\ ) 
\frac{1}{\sqrt{30}}\begin{pmatrix}
1\\-2\\5
\end{pmatrix} 
 \Bigg|^2 \\
& = \Big | 
\frac{1}{\sqrt{30}} 
  \Big |^2 \\
  \mathscr{P}_{a_1} & = \frac{1}{30}
\end{align*}
\begin{align*}
\mathscr{P}_{a_2} & = \bigg| \braket{a_2|\psi} \bigg|^2 \\
& = \Bigg| (\ 0\quad 1\quad 0\ ) 
\frac{1}{\sqrt{30}}\begin{pmatrix}
1\\-2\\5
\end{pmatrix} 
 \Bigg|^2 \\
& = \Big | 
\frac{-2}{\sqrt{30}} 
  \Big |^2 \\
  \mathscr{P}_{a_2} & = \frac{4}{30}
\end{align*}
\begin{align*}
\mathscr{P}_{a_3} & = \bigg| \braket{a_3|\psi} \bigg|^2 \\
& = \Bigg| (\ 0\quad 0\quad 1\ ) 
\frac{1}{\sqrt{30}}\begin{pmatrix}
1\\-2\\5
\end{pmatrix} 
 \Bigg|^2 \\
& = \Big | 
\frac{5}{\sqrt{30}} 
  \Big |^2 \\
  \mathscr{P}_{a_3  } & = \frac{25}{30} = \frac{5}{6}
\end{align*}
\end{multicols}
\begin{center}
\begin{tikzpicture}
    \begin{axis}[
      ybar,
      bar width=15pt,
      xlabel={$\bra{\psi}$},
      ylabel={Probability},
      ymin=0,
      ymax =100,
      ytick=\empty,
      xtick=data,
      axis x line=bottom,
      axis y line=left,
      enlarge x limits=0.2,
      symbolic x coords={ $ a_1$, $ a_2$, $a_3$ },
      xticklabel style={anchor=base,yshift=-\baselineskip},
      nodes near coords={\pgfmathprintnumber\pgfplotspointmeta\%}
    ]
      \addplot[fill=white] coordinates {
        ($ a_1$,3.33)
        ($ a_2$,13.3)
        ($a_3$, 83.3)
      };
    \end{axis}
  \end{tikzpicture}
\end{center}
\textbf{c) In a different experiment, the system is prepared in the state}\\
$$
\ket{\psi} = 2\ket{a_1} +3i\ket{a_2}
$$
\textbf{Write this state in matrix notation and calculate the probabilities of all possible measurement
results of the observable A. Plot a histogram of the predicted measurement results.}
\begin{align*}
\ket{\psi} & = 
\begin{pmatrix}
2\\3i\\0
\end{pmatrix} \quad \text{Normalize the function} \\
\braket{\psi|\psi} & = C^* (\ 2\quad -3i\quad 0\ )\ 
C \begin{pmatrix}
2\\ 3i\\ 0
\end{pmatrix}\\
& = C^2 (4 + 9)\\
C & = \frac{1}{\sqrt{13}} \\
\ket{\psi} & = \frac{1}{\sqrt{13}}
\begin{pmatrix}
2\\3i \\ 0
\end{pmatrix}
\end{align*}

\begin{multicols}{3}
\noindent
\begin{align*}
\mathscr{P}_{a_1} & = \bigg| \braket{a_1|\psi} \bigg|^2 \\
& = \Bigg| (\ 1\quad 0\quad 0\ ) 
\frac{1}{\sqrt{13}}\begin{pmatrix}
2\\3i\\0
\end{pmatrix} 
 \Bigg|^2 \\
& = \Big | 
\frac{2}{\sqrt{13}} 
  \Big |^2 \\
  \mathscr{P}_{a_1} & = \frac{4}{13}
\end{align*}
\begin{align*}
\mathscr{P}_{a_2} & = \bigg| \braket{a_2|\psi} \bigg|^2 \\
& = \Bigg| (\ 0\quad 1\quad 0\ ) 
\frac{1}{\sqrt{13}}\begin{pmatrix}
2\\3i\\0
\end{pmatrix} 
 \Bigg|^2 \\
& = \Big | 
\frac{3i}{\sqrt{13}} 
  \Big |^2 \\
  \mathscr{P}_{a_2} & = \frac{9}{13}
\end{align*}
\begin{align*}
\mathscr{P}_{a_3} & = \bigg| \braket{a_3|\psi} \bigg|^2 \\
& = \Bigg| (\ 0\quad 0\quad 1\ ) 
\frac{1}{\sqrt{13}}\begin{pmatrix}
2\\3i\\0
\end{pmatrix} 
 \Bigg|^2 \\
& = 0 \\
  \mathscr{P}_{a_3} & = 0
\end{align*}
\end{multicols}



\begin{center}
\begin{tikzpicture}
    \begin{axis}[
      ybar,
      bar width=15pt,
      xlabel={$\bra{\psi}$},
      ylabel={Probability},
      ymin=0,
      ymax =100,
      ytick=\empty,
      xtick=data,
      axis x line=bottom,
      axis y line=left,
      enlarge x limits=0.2,
      symbolic x coords={ $ a_1$, $ a_2$, $a_3$ },
      xticklabel style={anchor=base,yshift=-\baselineskip},
      nodes near coords={\pgfmathprintnumber\pgfplotspointmeta\%}
    ]
      \addplot[fill=white] coordinates {
        ($ a_1$,30.7)
        ($ a_2$,69.2)
        ($a_3$, 0)
      };
    \end{axis}
  \end{tikzpicture}
\end{center}
 %%%%%%%%%%%%%%%%%%%%%%%%%%%%%%%%%%%%%%%%%%%%%%%%%%%%%%%%%%%%%%%%%%%%%%%%%%%%%%%%%%%%
\section*{Problem 1.15}
\textbf{Consider a quantum system described by a basis} $\ket{a_1}, \ket{a_2},$ and $\ket{a_3}$. \textbf{The system is initially in a state}\\
\begin{equation*}
\ket{\psi_i} = \tfrac{i}{\sqrt{3}} \ket{a_1} + \sqrt{\tfrac{2}{3}} \ket{a_2}.
\end{equation*}
\textbf{Find the probability that the system is measured to be in the final state}
\begin{equation*}
\ket{\psi_f} = \tfrac{1+i}{\sqrt{3}} \ket{a_1} + \frac{1}{\sqrt{6}} \ket{a_2} +\tfrac{1}{\sqrt{6}} \ket{a_3}.
\end{equation*}
\begin{align*}
\mathscr{P}_{\psi_f} & = \Big |\braket{\psi_f|\psi_i} \bigg|^2 \\
& = \Bigg|\bigg(\tfrac{1-i}{\sqrt{3}}\quad \tfrac{1}{\sqrt{6}}\quad \tfrac{1}{\sqrt{6}} \bigg)  
\begin{pmatrix}
\tfrac{i}{\sqrt{3}} \\
\sqrt{\tfrac{2}{3}} \\
0
\end{pmatrix}
 \Bigg|^2 \\
 & =  \Bigg |  \frac{1-i}{3} + \sqrt{\frac{2}{9}} + 0  \Bigg |^2 \\
 & =  \Bigg |  \frac{1-i}{3} + \frac{1}{3}  \Bigg |^2 \\
 & = \Bigg (  \frac{2}{3} + \frac{1}{3}  \Bigg )^2 \\
 \mathscr{P}_{\psi_f} & = \frac{5}{9} 
\end{align*}
%----------------------------------------------------------------------------------------
\section*{Problem 1.16}
\textbf{The spin components of a beam of atoms prepared in the state} $\ket{\psi_{in}}$ \textbf{are measured and the following
experimental probabilities are obtained:}
\begin{multicols}{3}
\noindent
\begin{align*}
\mathscr{P}_+ & = \tfrac{1}{2} \\
\mathscr{P}_- & = \tfrac{1}{2}
\end{align*}
\begin{align*}
\mathscr{P}_{+x} & = \tfrac{3}{4} \\
\mathscr{P}_{-x} & = \tfrac{1}{4}
\end{align*}
\begin{align*}
\mathscr{P}_{+y} & = 0.067 \\
\mathscr{P}_{-y} & = 0.933
\end{align*}
\end{multicols}
\textbf{From the experimental data, determine the input state.}
\begin{align*}
\frac{1}{2} & = \big | \braket{+|\psi} \big |^2  &  \frac{1}{2} & = \big | \braket{-|\psi} \big |^2                                    \\
& = \bigg | \bra{+} \bigg \{ a\ket{+} + b\ket{-} \Bigg \} \bigg|^2 &    & = \bigg | \bra{-} \bigg \{ a\ket{+} + b\ket{-} \Bigg \} \bigg|^2                   \\
& = \bigg | a \braket{+|+} \bigg|^2         &        & = \bigg | b \braket{-|-} \bigg|^2                                      \\
& = \bigg | a \bigg |^2       &         & = \bigg | b \bigg |^2                                                    \\
a & = \frac{1}{\sqrt{2}}  &   b & = \frac{1}{\sqrt{2}}
\end{align*}
\horrule{2pt}
\begin{multicols}{2}
\noindent
\begin{align*}
\tfrac{3}{4} & = \big | \braket{+_x| \psi}  \big |^2 \\
\tfrac{3}{4} & = \bigg |  \tfrac{1}{\sqrt{2}} \bigg ( \bra{+} + \bra{-} \bigg ) \tfrac{1}{\sqrt{2}} \bigg ( \ket{+}  + e^{i\phi}\ket{-} \bigg )  \bigg |^2 \\
& = \bigg | \tfrac{1}{2} \big ( 1 + e^{i\phi}  \big )  \bigg|^2 \\
& = \tfrac{1}{4} \big ( 1 + e^{i\phi}  \big ) \big ( 1 + e^{-i\phi}   \big ) \\
& = \tfrac{1}{4} \big (  1^2 + 2\cos(\phi) + e^0 \big ) \\
& = \tfrac{1}{4} \big ( 2 + 2\cos(\phi) \big ) \\
& + \tfrac{1}{2} \big (  1 + \cos(\phi) \big ) \\
\tfrac{3}{2} - 1 & = \cos(\phi) \\
\cos^{-1}(\tfrac{1}{2}) &= \pm \tfrac{\pi}{3}
\end{align*}
\begin{align*}
\tfrac{1}{4} & = \big | \braket{-_x| \psi}  \big |^2 \\
\tfrac{1}{4} & = \bigg |  \tfrac{1}{\sqrt{2}} \bigg ( \bra{+} - \bra{-} \bigg ) \tfrac{1}{\sqrt{2}} \bigg ( \ket{+}  + e^{i\phi}\ket{-} \bigg )  \bigg |^2 \\
& = \bigg | \tfrac{1}{2} \big ( 1 - e^{i\phi}  \big )  \bigg|^2 \\
& = \tfrac{1}{4} \big ( 1 - e^{i\phi}  \big ) \big ( 1 - e^{-i\phi}   \big ) \\
& = \tfrac{1}{4} \big (  1^2 - 2\cos(\phi) + e^0 \big ) \\
& = \tfrac{1}{4} \big ( 2 - 2\cos(\phi) \big ) \\
& = \tfrac{1}{2} \big (  1 - \cos(\phi) \big ) \\
\cos(\phi) & = 1- \tfrac{1}{2}  \\
\cos^{-1}(\tfrac{1}{2}) & =  \pm \tfrac{\pi}{3}
\end{align*}
\end{multicols}
\horrule{2pt}
\begin{multicols}{2}
\noindent
\begin{align*}
0.067 & = \big | \braket{+_y| \psi}  \big |^2 \\
 & = \bigg |  \tfrac{1}{\sqrt{2}} \bigg ( \bra{+} -i \bra{-} \bigg ) \tfrac{1}{\sqrt{2}} \bigg ( \ket{+}  + e^{i\phi}\ket{-} \bigg )  \bigg |^2 \\
& = \bigg | \tfrac{1}{2} \big ( 1 - ie^{i\phi}  \big )  \bigg|^2 \\
& = \tfrac{1}{4} \big ( 1 - ie^{i\phi}  \big ) \big ( 1 + ie^{-i\phi}   \big ) \\
& = \tfrac{1}{4} \big (  1^2 + 2\sin(\phi) + e^0 \big ) \\
& = \tfrac{1}{4} \big ( 1+ 2\sin(\phi) \big ) \\
& = \tfrac{1}{2} \big (  1 + \sin(\phi) \big ) \\
\sin(\phi) & = 2(0.067) - 1 \\
\sin^{-1}(-0.866) & = - \tfrac{\pi}{3 } \ or\ -\tfrac{2\pi}{3}
\end{align*}
\begin{align*}
0.933 & = \big | \braket{+_y| \psi}  \big |^2 \\
 & = \big |  \tfrac{1}{\sqrt{2}} \big ( \bra{+} +i \bra{-} \bigg ) \tfrac{1}{\sqrt{2}} \big ( \ket{+}  + e^{i\phi}\ket{-} \big )  \bigg |^2 \\
& = \bigg | \tfrac{1}{2} \big ( 1 + ie^{i\phi}  \big )  \bigg|^2 \\
& = \tfrac{1}{4} \big ( 1 + ie^{i\phi}  \big ) \big ( 1 - ie^{-i\phi}   \big ) \\
& = \tfrac{1}{4} \big (  1^2 + 2\sin(\phi) + e^0 \big ) \\
& = \tfrac{1}{4} \big ( 1+ 2\sin(\phi) \big ) \\
& = \tfrac{1}{2} \big (  1 + \sin(\phi) \big ) \\
\sin(\phi) & = 2(0.933) - 1 \\
\sin^{-1}(0.866) & = \pm \tfrac{\pi}{3 } \ 
\end{align*}
\end{multicols}
\begin{equation*}
\ket{\psi_{in}} = \tfrac{1}{\sqrt{2}} \big ( \ket{+} + e^{-i\tfrac{\pi}{3}} \ket{-} \big )
\end{equation*}
\end{document}